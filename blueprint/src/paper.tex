\documentclass[11pt]{article}

\usepackage{amsmath, amsthm, amssymb}
\usepackage{hyperref}
\usepackage{geometry}
\geometry{margin=1in}

\theoremstyle{definition}
\newtheorem{definition}{Definition}[section]
\newtheorem{theorem}{Theorem}[section]
\newtheorem{lemma}{Lemma}[section]
\newtheorem{corollary}{Corollary}[section]
\newtheorem{remark}{Remark}[section]
\newtheorem{example}{Example}[section]

\title{The Crystallographic Restriction Theorem:\\A Formalized Proof in Lean 4}
\author{Eric Vergo \and Claude (Anthropic)}

\newcommand{\Z}{\mathbb{Z}}
\newcommand{\Q}{\mathbb{Q}}
\newcommand{\N}{\mathbb{N}}
\newcommand{\R}{\mathbb{R}}
\newcommand{\ord}{\mathrm{ord}}
\newcommand{\lcm}{\mathrm{lcm}}
\newcommand{\GL}{\mathrm{GL}}

\begin{document}
\maketitle

\begin{abstract}
We present a complete formalization in the Lean 4 proof assistant of the crystallographic
restriction theorem, which characterizes exactly which finite orders can be achieved by integer matrices of a given dimension. The theorem states that an $N \times N$ matrix with integer
entries can have finite order $m$ if and only if $\psi(m) \leq N$, where $\psi$ is a function closely related to Euler's totient function. This elegant result connects linear algebra over the integers
with number-theoretic properties and explains why crystal lattices can only exhibit 2-, 3-, 4-,
or 6-fold rotational symmetry. Our formalization follows the exposition of Kuzmanovich and
Pavlichenkov, establishing both the forward direction (necessity) via minimal polynomial theory
and the backward direction (sufficiency) via explicit construction using companion matrices of
cyclotomic polynomials.
\end{abstract}

\section{Introduction}

Finite groups of matrices with integer entries arise naturally in many areas of mathematics and
physics. In crystallography, the symmetry groups of crystal lattices are realized as finite subgroups
of $\GL(n,\Z)$, the group of $n \times n$ matrices with integer entries whose inverses also have integer entries.
A fundamental question is: \emph{which finite orders can elements of $\GL(n,\Z)$ have?}

The answer involves a beautiful interplay between linear algebra, number theory, and algebra.
Minkowski proved the remarkable result that $\GL(n, \Z)$ contains only finitely many isomorphism
classes of finite subgroups, which implies that there are only finitely many possible orders for
elements of finite order. The crystallographic restriction theorem provides a precise characterization
of these orders.

\subsection{Historical Context}

The name ``crystallographic restriction'' originates from the physical study of crystals. A crystal
lattice in $\R^n$ is a discrete subgroup of translations, and the rotational symmetries of such a lattice
must preserve the lattice structure. This requirement severely constrains the possible rotation
angles.

In two and three dimensions, the classical crystallographic restriction states that a rotation
preserving a lattice must have order 1, 2, 3, 4, or 6. This explains why crystals can have 2-fold,
3-fold, 4-fold, or 6-fold rotational symmetry, but never 5-fold or 7-fold symmetry---a fact observed
experimentally long before it was proved mathematically.

The general theorem we formalize extends this to all dimensions, answering completely which
orders are achievable in $\GL(N, \Z)$ for any $N$.

\subsection{Statement of the Main Result}

The characterization involves a function $\psi : \N \to \N$ that we call the \emph{dimensional cost function}.
Informally, $\psi(m)$ measures the minimum dimension needed to construct an integer matrix of order $m$.

\paperstatement{thm:main-theorem}

The function $\psi$ is defined as follows. For a prime power $p^k$:
$$\psi_{\text{pp}}(p, k) = \begin{cases}
0 & \text{if } k = 0, \\
0 & \text{if } p = 2 \text{ and } k = 1, \\
\varphi(p^k) & \text{otherwise},
\end{cases}$$
where $\varphi$ denotes Euler's totient function. For a general positive integer $m = \prod_i p_i^{k_i}$, we define
$\psi(m) = \sum_i \psi_{\text{pp}}(p_i, k_i)$.

The special treatment of $2^1$ reflects the fact that order 2 is ``free'' in any positive dimension:
the matrix $-I$ has order 2 in any dimension $N \geq 1$.

\subsection{Significance and Applications}

The crystallographic restriction has several notable consequences:

\begin{enumerate}
\item \textbf{Classical 2D/3D restriction:} Since $\psi(m) \leq 2$ only for $m \in \{1, 2, 3, 4, 6\}$, these are the
only achievable orders in dimensions 2 and 3, explaining the observed rotational symmetries
of crystals.

\item \textbf{Dimension parity:} Since $\psi(m)$ is always even for $m > 2$, the achievable orders in dimension
$2k$ are the same as in dimension $2k + 1$. This surprising result means that odd dimensions
offer no new orders beyond their even predecessors.

\item \textbf{First occurrence of orders:} The theorem determines exactly when each order first becomes
achievable. For instance, order 5 requires dimension at least 4 (since $\psi(5) = \varphi(5) = 4$), while
order 7 requires dimension at least 6.
\end{enumerate}

\subsection{Overview of the Proof}

Our formalization follows the proof structure of Kuzmanovich and Pavlichenkov~\cite{KP2002}. The proof
proceeds in two directions:

\textbf{Forward direction (necessity):} If an $N \times N$ integer matrix $A$ has order $m$, we show $\psi(m) \leq N$. The key insight is that the minimal polynomial of $A$ (viewed over $\Q$) must be a product of
distinct cyclotomic polynomials $\Phi_d$ for various divisors $d$ of $m$. The constraint that the order is
exactly $m$ forces the lcm of these divisors to equal $m$, leading to the dimension bound.

\textbf{Backward direction (sufficiency):} If $\psi(m) \leq N$, we construct an explicit $N \times N$ integer matrix with order $m$. The construction uses companion matrices of cyclotomic polynomials combined
via block diagonal matrices.

The formalization required developing substantial supporting infrastructure, including properties of cyclotomic polynomials, companion matrices, block diagonal constructions, and the $\psi$
function itself.

\section{Preliminaries}

We establish notation and recall the key definitions needed for the proof.

\subsection{Notation and Conventions}

Throughout, we use the following notation:
\begin{itemize}
\item $\N$ denotes the natural numbers $\{0, 1, 2, \ldots\}$
\item $\Z$ denotes the integers
\item $\Q$ denotes the rational numbers
\item $\GL(N, \Z)$ denotes the group of $N \times N$ invertible integer matrices
\item $\varphi(m)$ denotes Euler's totient function
\item $\Phi_m(X)$ denotes the $m$-th cyclotomic polynomial
\item $\ord(A)$ denotes the multiplicative order of a matrix $A$ (the smallest positive $k$ with $A^k = I$)
\end{itemize}

For a matrix $A$ with $A^m = I$ for some $m > 0$, we say $A$ has \emph{finite order}, and $\ord(A)$ is the
smallest such $m$.

We write $p^k \| m$ to mean that $p^k$ is the exact power of $p$ dividing $m$, that is, $p^k \mid m$ but $p^{k+1} \nmid m$.

\subsection{The Psi Function}

The dimensional cost function $\psi$ is central to the crystallographic restriction. We first define it on
prime powers.

\paperstatement{psiPrimePow-def}

The function $\psi_{\text{pp}}(p, k)$ equals $\varphi(p^k) = (p-1)p^{k-1}$ in most cases, with two exceptions: $\psi_{\text{pp}}(p, 0) = 0$ for any prime $p$, and $\psi_{\text{pp}}(2, 1) = 0$. The latter exception captures the fact that order 2 requires
no ``dimensional cost'' since $-I$ has order 2 in any positive dimension.

\paperstatement{psi-def}

\subsection{Properties of Psi}

Several properties of $\psi$ are essential for the proof.

\begin{lemma}[Explicit Values]
For small values of $m$:
\begin{center}
\begin{tabular}{c|cccccccccccc}
$m$ & 1 & 2 & 3 & 4 & 5 & 6 & 7 & 8 & 9 & 10 & 11 & 12 \\
\hline
$\psi(m)$ & 0 & 0 & 2 & 2 & 4 & 2 & 6 & 4 & 6 & 4 & 10 & 4
\end{tabular}
\end{center}
\end{lemma}

\paperfull{lem:psi-prime-pow}

The power of $\psi$ comes from its additivity on coprime factors.

\paperfull{lem:psi-coprime-add}

The function $\psi$ is bounded above by the totient function.

\paperfull{lem:psi-le-totient}

\begin{remark}
The only case where $\psi(m) < \varphi(m)$ is when $2 \| m$ (that is, $m$ is divisible by 2 but not
by 4). In this case, the factor of 2 contributes $\varphi(2) = 1$ to the totient but 0 to $\psi$.
\end{remark}

\section{Companion Matrices}

Companion matrices are the fundamental tool for constructing matrices with prescribed characteristic polynomials. They play a central role in the backward direction of our proof.

\subsection{Definition and Basic Properties}

\paperstatement{companion-def}

The companion matrix of a monic polynomial provides a canonical matrix realization
whose characteristic polynomial equals the original polynomial.

\paperfull{thm:companion-charpoly}

By the Cayley-Hamilton theorem, every matrix satisfies its characteristic polynomial. For
companion matrices, this means the defining polynomial evaluates to zero at the companion matrix.

\paperfull{lem:companion-aeval-zero}

\paperfull{thm:companion-pow-dvd}

\subsection{Cyclotomic Companion Matrices}

When we apply the companion matrix construction to cyclotomic polynomials, we obtain integer
matrices with precisely controlled finite orders.

Recall that the $m$-th cyclotomic polynomial $\Phi_m(X)$ is the minimal polynomial over $\Q$ of a
primitive $m$-th root of unity. It has three crucial properties:
\begin{enumerate}
\item $\Phi_m(X)$ has integer coefficients.
\item $\Phi_m(X)$ has degree $\varphi(m)$.
\item $\Phi_m(X)$ divides $X^m - 1$ but does not divide $X^k - 1$ for any $0 < k < m$.
\end{enumerate}

\paperfull{lem:companion-cycl-pow}

\paperfull{thm:companion-cycl-order}

\paperfull{thm:companion-cycl-mem}

This gives us the key result for constructing matrices of prescribed order.

\paperfull{thm:mem-orders-totient}

\begin{example}
The cyclotomic polynomial $\Phi_3(X) = X^2 + X + 1$ has companion matrix
$$C(\Phi_3) = \begin{pmatrix} 0 & -1 \\ 1 & -1 \end{pmatrix}.$$
One can verify directly: $C(\Phi_3)^2 = \begin{pmatrix} -1 & 1 \\ -1 & 0 \end{pmatrix}$ and $C(\Phi_3)^3 = I$. This $2 \times 2$ integer matrix has order exactly 3.
\end{example}

\begin{example}
The cyclotomic polynomial $\Phi_6(X) = X^2 - X + 1$ has companion matrix
$$C(\Phi_6) = \begin{pmatrix} 0 & -1 \\ 1 & 1 \end{pmatrix}.$$
This $2 \times 2$ matrix has order 6, demonstrating that order 6 is achievable in dimension 2.
\end{example}

\section{Block Diagonal Matrices}

To construct matrices of arbitrary orders from companion matrices of cyclotomic polynomials, we
use block diagonal combinations.

\subsection{Definition}

\paperstatement{def:blockDiag2}

This construction extends naturally to any finite number of blocks.

\subsection{Order of Block Diagonal Matrices}

\paperfull{lem:blockDiag2-one}

\paperfull{lem:blockDiag2-mul}

\paperfull{lem:blockDiag2-pow}

\paperfull{thm:orderOf-blockDiag2}

\begin{corollary}
If $A$ has order $a$, $B$ has order $b$, and $\gcd(a, b) = 1$, then $\text{diag}(A, B)$ has order $ab$.
\end{corollary}

\paperfull{lem:lcm-mem-orders}

\section{The Forward Direction}

We now prove that if an integer matrix has order $m$, then its dimension must be at least $\psi(m)$.

\subsection{Minimal Polynomials of Matrices with Finite Order}

Let $A \in \GL(N, \Z)$ be a matrix with finite order $m$. Since $A^m = I$, the polynomial $X^m - 1$
annihilates $A$. The minimal polynomial $\mu_A$ of $A$ (over $\Q$) therefore divides $X^m - 1$.

\begin{lemma}
Over $\Q$, we have the factorization
$$X^m - 1 = \prod_{d \mid m} \Phi_d(X),$$
where the product is over all positive divisors $d$ of $m$.
\end{lemma}

This is a standard result: every $m$-th root of unity is a primitive $d$-th root of unity for exactly
one divisor $d$ of $m$.

\paperfull{lem:pow-eq-one-of-minpoly-dvd}

\paperfull{lem:minpoly-dvd-X-pow-sub-one}

\subsection{The Order Constraint}

\paperfull{lem:cyclotomic-finset-product-dvd}

\paperfull{lem:minpoly-dvd-prod-cyclotomic}

\paperfull{lem:minpoly-eq-prod-cyclotomic}

\paperfull{lem:cyclotomic-divisors-lcm-eq}

\subsection{The Key Inequality}

The heart of the forward direction is the following combinatorial lemma about subsets of divisors.

\paperfull{lem:sum-totient-ge-psi}

\subsection{Proof of the Forward Direction}

\paperfull{thm:forward-direction}

\section{The Backward Direction}

We now show that every order satisfying the necessary condition is actually achievable.

\subsection{Prime Power Orders}

By Corollary~3.7, for any prime power $p^k$ with $k \geq 1$, the companion matrix $C(\Phi_{p^k})$ achieves order
$p^k$ in dimension $\varphi(p^k)$.

\paperfull{thm:primePow-mem-integerMatrixOrders-psi}

\subsection{Order 2 is Free}

The special case $(p, k) = (2, 1)$ requires separate treatment.

\paperfull{lem:two-mem-orders}

This is why $\psi(2) = 0$: achieving order 2 costs nothing beyond having at least one dimension.

\paperfull{lem:orderOf-neg-of-odd-order}

\begin{corollary}
If order $k$ (odd) is achievable in dimension $N$, then order $2k$ is achievable in
dimension $N$.
\end{corollary}

\subsection{General Construction}

\paperfull{thm:mem-integerMatrixOrders-psi}

\paperfull{thm:backward-direction}

\begin{example}[Order 12 in Dimension 4]
We have $12 = 4 \cdot 3 = 2^2 \cdot 3$, so we use Case 3 with
$a = 2$ and $m' = 3$.

For the factor $2^2 = 4$: $C(\Phi_4) = C(X^2 + 1) = \begin{pmatrix} 0 & -1 \\ 1 & 0 \end{pmatrix}$, which has order 4 and dimension $\varphi(4) = 2$.

For the factor 3: $C(\Phi_3) = \begin{pmatrix} 0 & -1 \\ 1 & -1 \end{pmatrix}$, which has order 3 and dimension $\varphi(3) = 2$.

The block diagonal
$$\text{diag}(C(\Phi_4), C(\Phi_3)) = \begin{pmatrix}
0 & -1 & 0 & 0 \\
1 & 0 & 0 & 0 \\
0 & 0 & 0 & -1 \\
0 & 0 & 1 & -1
\end{pmatrix}$$
has order $\lcm(4, 3) = 12$ and dimension $2 + 2 = 4 = \psi(12)$.
\end{example}

\section{The Crystallographic Restriction Theorem}

We now combine the forward and backward directions.

\subsection{The Main Result}

\paperfull{thm:main-theorem}

\paperstatement{integerMatrixOrders-def}

\begin{corollary}
$\text{Ord}_N = \{m \geq 1 : \psi(m) \leq N\}$.
\end{corollary}

\subsection{The Classical Crystallographic Restriction}

\begin{corollary}[Classical Restriction]
In dimensions 2 and 3, the achievable orders are exactly
$\{1, 2, 3, 4, 6\}$.
\end{corollary}

\begin{proof}
From the explicit values table, $\psi(m) \leq 2$ if and only if $m \in \{1, 2, 3, 4, 6\}$:
\begin{itemize}
\item $\psi(1) = \psi(2) = 0 \leq 2$ \checkmark
\item $\psi(3) = \psi(4) = \psi(6) = 2 \leq 2$ \checkmark
\item $\psi(5) = 4 > 2$ \texttimes
\item $\psi(7) = 6 > 2$ \texttimes
\end{itemize}
Since $\psi(m) \leq 3$ if and only if $\psi(m) \leq 2$ (as $\psi(m)$ is always even for $m > 2$), the achievable orders
in dimensions 2 and 3 coincide.
\end{proof}

This explains why crystals in our three-dimensional world exhibit only 2-, 3-, 4-, and 6-fold
rotational symmetries: these correspond to the elements of orders 2, 3, 4, and 6 in $\GL(3, \Z)$.

\subsection{Dimension Parity}

\begin{corollary}
For $k \geq 1$, the sets $\text{Ord}_{2k}$ and $\text{Ord}_{2k+1}$ are equal.
\end{corollary}

\begin{proof}
For $m > 2$, the value $\psi(m) = \sum_{(p,k) \neq (2,1)} \varphi(p^k)$ is a sum of terms $\varphi(p^k) = (p-1)p^{k-1}$. Each
such term is even: if $p$ is odd, then $p - 1$ is even; if $p = 2$ and $k \geq 2$, then $p^{k-1} = 2^{k-1}$ is even.
Thus $\psi(m)$ is even for all $m \geq 1$. The condition $\psi(m) \leq 2k$ is equivalent to $\psi(m) \leq 2k + 1$.
\end{proof}

This means that passing from an even dimension to the next odd dimension never unlocks new
achievable orders.

\subsection{First Occurrence Table}

The following table shows when each small order first becomes achievable:

\begin{center}
\begin{tabular}{c|cccccccccccc}
Order $m$ & 1 & 2 & 3 & 4 & 5 & 6 & 7 & 8 & 9 & 10 & 11 & 12 \\
\hline
$\psi(m)$ & 0 & 0 & 2 & 2 & 4 & 2 & 6 & 4 & 6 & 4 & 10 & 4 \\
First dimension & 1 & 1 & 2 & 2 & 4 & 2 & 6 & 4 & 6 & 4 & 10 & 4
\end{tabular}
\end{center}

We observe:
\begin{itemize}
\item Orders 1 and 2 are achievable in dimension 1 (the smallest possible).
\item Orders 3, 4, 6 first appear in dimension 2.
\item Orders 5, 8, 10, 12 first appear in dimension 4.
\item Orders 7, 9 first appear in dimension 6.
\item Order 11 requires dimension 10.
\end{itemize}

\begin{thebibliography}{9}

\bibitem{KP2002}
J. Kuzmanovich and A. Pavlichenkov.
Finite groups of matrices whose entries are integers.
\emph{The American Mathematical Monthly}, 109(2):173--186, 2002.

\bibitem{BCK2003}
J. Bamberg, G. Cairns, and D. Kilminster.
The crystallographic restriction, permutations, and Goldbach's conjecture.
\emph{The American Mathematical Monthly}, 110(3):202--209, 2003.

\bibitem{Newman1972}
M. Newman.
\emph{Integral Matrices}.
Academic Press, New York, 1972.

\bibitem{DF2004}
D. S. Dummit and R. M. Foote.
\emph{Abstract Algebra}.
Prentice Hall, 3rd edition, 2004.

\bibitem{Lang2002}
S. Lang.
\emph{Algebra}.
Springer, 3rd edition, 2002.

\end{thebibliography}

\end{document}
