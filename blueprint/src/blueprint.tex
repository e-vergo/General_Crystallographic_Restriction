\documentclass{article}

\input{../../.lake/build/dressed/library/Crystallographic.tex}

\usepackage{amsmath, amsthm, amssymb}
\usepackage{hyperref}

\theoremstyle{definition}
\newtheorem{definition}{Definition}[section]
\newtheorem{theorem}{Theorem}[section]
\newtheorem{lemma}{Lemma}[section]
\newtheorem{corollary}{Corollary}[section]

\title{The Crystallographic Restriction Theorem}
\author{Eric Vergo}

\newcommand{\Z}{\mathbb{Z}}
\newcommand{\Q}{\mathbb{Q}}
\newcommand{\N}{\mathbb{N}}
\newcommand{\Ord}{\mathrm{Ord}}
\newcommand{\ord}{\mathrm{ord}}
\newcommand{\lcm}{\mathrm{lcm}}

\begin{document}
\maketitle

\chapter{Companion Matrices}

Companion matrices provide a systematic way to construct matrices with prescribed characteristic
polynomials, making them the fundamental tool for achieving specific matrix orders.

\section{Basic Definitions}

Given a monic polynomial $p(X) = X^n + a_{n-1}X^{n-1} + \cdots + a_1 X + a_0$, its
\emph{companion matrix} is the $n \times n$ matrix:
$$C(p) = \begin{pmatrix}
0 & 0 & \cdots & 0 & -a_0 \\
1 & 0 & \cdots & 0 & -a_1 \\
0 & 1 & \cdots & 0 & -a_2 \\
\vdots & \vdots & \ddots & \vdots & \vdots \\
0 & 0 & \cdots & 1 & -a_{n-1}
\end{pmatrix}$$

The companion matrix of a monic polynomial provides a canonical matrix realization
whose characteristic polynomial equals the original polynomial.

\inputleannode{companion-def}

\section{Characteristic Polynomial}

The fundamental property of companion matrices is that their characteristic polynomial
equals the defining polynomial. This means we can ``design'' a matrix to have any
monic polynomial as its characteristic polynomial.

\inputleannode{thm:companion-charpoly}

Since the companion matrix satisfies its own characteristic polynomial (by Cayley-Hamilton),
evaluating the defining polynomial at the companion matrix gives zero.

\inputleannode{lem:companion-aeval-zero}

This immediately gives us control over powers of the companion matrix: if $p(X)$ divides
$X^m - 1$, then $C(p)^m = I$.

\inputleannode{thm:companion-pow-dvd}

\section{Cyclotomic Companion Matrices}

Companion matrices of cyclotomic polynomials have particularly nice properties:
they are integer matrices with prescribed finite order. This makes them the building
blocks for our backward direction construction.

The $m$-th cyclotomic polynomial $\Phi_m(X)$ has integer coefficients (a non-trivial fact),
degree $\varphi(m)$, and divides $X^m - 1$ but not $X^k - 1$ for any $k < m$. These
properties transfer directly to its companion matrix.

\inputleannode{lem:companion-cycl-pow}

The order of the companion matrix of $\Phi_m$ is exactly $m$. This is because $\Phi_m$
is the minimal polynomial of primitive $m$-th roots of unity, so $\Phi_m(X)$ divides
$X^m - 1$ but not $X^k - 1$ for $k < m$.

\inputleannode{thm:companion-cycl-order}

Since $\Phi_m$ has integer coefficients, the companion matrix has integer entries.

\inputleannode{thm:companion-cycl-mem}

This gives us the key result for constructing matrices of prescribed order: the companion
matrix of $\Phi_m$ achieves order $m$ in dimension $\varphi(m)$.

\inputleannode{thm:mem-orders-totient}

\chapter{The Psi Function}

The $\psi$ function is the key invariant that characterizes achievable matrix orders.
It refines Euler's totient function $\varphi$ by accounting for the special role of
order 2.

\section{Definition}

The function $\psi : \N \to \N$ computes the minimum dimension needed to achieve a
given order. The key insight is that order 2 is ``free'' in any positive dimension
(achieved by $-I$), so we should not count the $\varphi(2) = 1$ contribution from
a single factor of 2.

For a prime power $p^k$, we define:
$$\psi_{\text{pp}}(p, k) = \begin{cases}
0 & \text{if } k = 0 \\
0 & \text{if } p = 2 \text{ and } k = 1 \\
\varphi(p^k) & \text{otherwise}
\end{cases}$$

\inputleannode{psiPrimePow-def}

For a general positive integer $m$ with prime factorization $m = \prod_i p_i^{k_i}$,
we define:
$$\psi(m) = \sum_i \psi_{\text{pp}}(p_i, k_i)$$

\inputleannode{psi-def}

\section{Basic Properties}

For a prime power $p^k$ (with $k \geq 1$), $\psi(p^k)$ equals $\varphi(p^k)$ in all cases
except $\psi(2) = 0$.

\inputleannode{lem:psi-prime-pow}

The power of $\psi$ comes from its additivity on coprime factors. When $\gcd(m, n) = 1$,
the prime factorizations of $m$ and $n$ are disjoint.

\inputleannode{lem:factorization-disjoint}

This disjointness gives us the key additivity property:

\inputleannode{lem:psi-coprime-add}

The function $\psi$ is bounded below by the contribution from any single prime power factor:

\inputleannode{lem:psi-ge-psiPrimePow}

\section{Bounds}

These bounds on the $\psi$ function are essential for the forward direction of the main theorem.
They establish relationships between $\psi$ and the totient function that allow us to
translate eigenvalue constraints into dimension bounds.

\inputleannode{lem:two-le-totient-prime-pow}

The following lemma provides a way to factor composite numbers that are not prime powers,
which is crucial for the inductive structure of our proofs.

\inputleannode{lem:factorization-split-lt}

For odd numbers at least 3, $\psi$ is strictly positive:

\inputleannode{lem:psi-pos-of-odd}

The function $\psi$ is bounded above by the totient function:

\inputleannode{lem:psi-le-totient}

The following lemmas establish key properties about sets of divisors whose lcm equals a
given value. These are essential for the forward direction, where we need to show that
if divisors $d_1, \ldots, d_s$ have $\lcm(d_1, \ldots, d_s) = m$, then the sum of their
totients is at least $\psi(m)$.

\inputleannode{lem:prime-pow-achieved-of-lcm-eq}

\inputleannode{lem:finset-nonempty-of-two-le-lcm}

\inputleannode{lem:finset-exists-one-lt-of-two-le-lcm}

The crucial bound relating sums of totients to $\psi$:

\inputleannode{lem:sum-totient-ge-psi}

\chapter{Integer Matrix Orders}

This chapter develops the theory of finite orders achievable by integer matrices,
including closure properties under block diagonal operations that are essential for
the backward direction.

\section{Basic Properties}

We define $\Ord_N$ as the set of finite orders achievable by $N \times N$ integer matrices:
$$\Ord_N = \{ m \in \N^+ : \exists A \in M_N(\Z),\ \ord(A) = m \}$$

\inputleannode{integerMatrixOrders-def}

The identity matrix has order 1, so $1 \in \Ord_N$ for all $N$:

\inputleannode{lem:one-mem-orders}

For $N \geq 1$, the matrix $-I$ has order 2:

\inputleannode{lem:two-mem-orders}

The set of achievable orders is monotone in dimension: if $m \in \Ord_N$ and $N \leq N'$,
then $m \in \Ord_{N'}$ (by padding with identity blocks).

\inputleannode{lem:orders-mono}

\section{Block Diagonal Matrices}

Block diagonal matrices are essential for combining matrices with coprime orders.
If $A$ has order $m$ and $B$ has order $n$ with $\gcd(m, n) = 1$, then the block
diagonal matrix $\begin{pmatrix} A & 0 \\ 0 & B \end{pmatrix}$ has order $mn$.

\inputleannode{def:blockDiag2}

Block diagonal respects the identity:

\inputleannode{lem:blockDiag2-one}

Block diagonal respects multiplication:

\inputleannode{lem:blockDiag2-mul}

This gives us a monoid homomorphism structure:

\inputleannode{def:blockDiag2-prodMonoidHom}

A block diagonal matrix equals the identity iff both blocks equal the identity:

\inputleannode{lem:blockDiag2-eq-one}

Powers distribute over block diagonal:

\inputleannode{lem:blockDiag2-pow}

\section{Order Results}

The order of a block diagonal matrix is the lcm of the block orders:

\inputleannode{thm:orderOf-blockDiag2}

This gives us the key closure property: $\Ord_N$ is closed under lcm for dimensions
that accommodate both factors.

\inputleannode{lem:lcm-mem-orders}

For coprime orders, the lcm equals the product, so we can combine coprime orders:

\inputleannode{lem:mul-mem-orders-coprime}

\chapter{The Crystallographic Restriction}

This chapter contains the forward and backward directions of the crystallographic restriction theorem.
Together, they establish that $m \in \Ord_N$ if and only if $\psi(m) \leq N$.

\section{Forward Direction}

The forward direction shows that if an $N \times N$ integer matrix has order $m$, then $\psi(m) \leq N$.
The proof uses eigenvalue theory: the minimal polynomial must be a product of cyclotomic polynomials,
and the constraint that the order is exactly $m$ (not some smaller divisor) forces enough cyclotomic
factors to appear that their degrees sum to at least $\psi(m)$.

The first lemma transfers polynomial divisibility to matrix powers:

\inputleannode{lem:pow-eq-one-of-minpoly-dvd}

Conversely, if $A^m = I$ then the minimal polynomial divides $X^m - 1$:

\inputleannode{lem:minpoly-dvd-X-pow-sub-one}

When cyclotomic polynomials divide a target, so does their product (since distinct
cyclotomic polynomials are coprime over $\Q$):

\inputleannode{lem:cyclotomic-finset-product-dvd}

The minimal polynomial divides the product of just those cyclotomic polynomials that divide it:

\inputleannode{lem:minpoly-dvd-prod-cyclotomic}

In fact, the minimal polynomial equals this product (by mutual divisibility of monic polynomials):

\inputleannode{lem:minpoly-eq-prod-cyclotomic}

The key constraint: if $\ord(A) = m$ and $\mu_A = \prod_{d \in S} \Phi_d$, then $\lcm(S) = m$.
Otherwise, $A^{\lcm(S)} = I$ with $\lcm(S) < m$, contradicting $\ord(A) = m$.

\inputleannode{lem:cyclotomic-divisors-lcm-eq}

The forward direction theorem:

\inputleannode{thm:forward-direction}

\section{Backward Direction}

The backward direction shows that if $\psi(m) \leq N$, then there exists an $N \times N$ integer matrix
with order $m$. The construction uses companion matrices of cyclotomic polynomials combined via
block diagonal.

First, we establish properties of permutation matrices, which provide an alternative construction
for some orders:

\inputleannode{lem:permMatrix-one}

\inputleannode{lem:permMatrix-mul}

\inputleannode{lem:permMatrix-pow}

\inputleannode{lem:permMatrix-eq-one-iff}

\inputleannode{lem:orderOf-permMatrix}

\inputleannode{lem:orderOf-finRotate}

\inputleannode{lem:orderOf-permMatrix-finRotate}

This gives us that $m \in \Ord_m$ for $m \geq 2$:

\inputleannode{lem:mem-integerMatrixOrders-self}

For prime powers (excluding $2^1$), we use cyclotomic companion matrices:

\inputleannode{thm:primePow-mem-integerMatrixOrders-psi}

The general case uses strong induction, combining the prime power result with
block diagonal constructions for composite $m$:

\inputleannode{thm:mem-integerMatrixOrders-psi}

The backward direction theorem:

\inputleannode{thm:backward-direction}

\chapter{Main Results}

This chapter contains the main theorem and its supporting lemmas, bringing together all
the machinery developed in previous chapters.

\section{Supporting Lemmas}

Several auxiliary results are needed for the final synthesis:

\inputleannode{lem:sum-le-prod}

\inputleannode{lem:lcm-factorization-le-sup}

\inputleannode{lem:primePow-mem-of-lcm-eq}

\inputleannode{lem:totient-ge-two}

\inputleannode{lem:prod-coprime-dvd}

\inputleannode{lem:totient-prod-coprime}

A key lemma for the construction: if $A$ has odd order $m$, then $-A$ has order $2m$:

\inputleannode{lem:orderOf-neg-of-odd-order}

\section{The Main Theorem}

We can now state and prove the crystallographic restriction theorem in its complete form.
The theorem provides a necessary and sufficient condition for an order to be achievable
by integer matrices of a given dimension.

\inputleannode{thm:main-theorem}

\section{Consequences}

The theorem has several immediate consequences:

\begin{enumerate}
\item \textbf{Classical 2D/3D restriction:} In dimensions 2 and 3, the achievable orders
      are $\{1, 2, 3, 4, 6\}$. This explains why crystals can only have 2-, 3-, 4-, or
      6-fold rotational symmetry.

\item \textbf{Dimension parity:} Since $\psi(m)$ is always even for $m > 2$, the achievable
      orders in dimension $2k$ are the same as in dimension $2k+1$.

\item \textbf{No 5-fold symmetry in low dimensions:} Since $\psi(5) = 4$, order 5 first
      becomes achievable in dimension 4.
\end{enumerate}

The following table shows $\psi(m)$ for small values of $m$:

\begin{center}
\begin{tabular}{c|cccccccccccc}
$m$ & 1 & 2 & 3 & 4 & 5 & 6 & 7 & 8 & 9 & 10 & 11 & 12 \\
\hline
$\psi(m)$ & 0 & 0 & 2 & 2 & 4 & 2 & 6 & 4 & 6 & 4 & 10 & 4
\end{tabular}
\end{center}

\end{document}
