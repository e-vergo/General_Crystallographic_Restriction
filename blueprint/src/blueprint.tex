\input{../../.lake/build/dressed/library/Crystallographic.tex}

\usepackage{amsmath, amsthm}
\usepackage{hyperref}

\theoremstyle{definition}
\newtheorem{definition}{Definition}[section]
\newtheorem{theorem}{Theorem}[section]
\newtheorem{proposition}{Proposition}[section]
\newtheorem{lemma}{Lemma}[section]
\newtheorem{corollary}{Corollary}[section]
\newtheorem{auxlemma}{Lemma}[section]

\title{Crystallographic Restriction Theorem}

\newcommand{\Z}{\mathbb{Z}}
\newcommand{\N}{\mathbb{N}}

\begin{document}
\maketitle

\chapter{Introduction}

The \emph{Crystallographic Restriction Theorem} characterizes exactly which orders can be
achieved by integer matrices of a given dimension. For an $N \times N$ matrix $M$ with
integer entries, we say $M$ has \emph{finite order} $m$ if $M^m = I$ and $m$ is the smallest
such positive integer. The theorem provides a complete answer to the question: which values
of $m$ can occur as orders of $N \times N$ integer matrices?

\medskip

\noindent\textbf{Main Theorem.} \textit{An $N \times N$ integer matrix can have finite order $m$
if and only if $\psi(m) \leq N$, where $\psi$ is defined below.}

\medskip

The function $\psi$ measures the ``arithmetic complexity'' of $m$ in terms of Euler's
totient function $\varphi$. For a prime power $p^k$, we define the \emph{prime power
contribution}:
\[
\psi_{\mathrm{pp}}(p, k) =
\begin{cases}
0 & \text{if } p = 2 \text{ and } k = 1, \\
\varphi(p^k) & \text{otherwise}.
\end{cases}
\]
The special case $\psi_{\mathrm{pp}}(2, 1) = 0$ reflects the fact that order 2 can be achieved
by the $0 \times 0$ empty matrix (or equivalently, $-I$ achieves order 2 without increasing
dimension). For a general positive integer $m$ with prime factorization
$m = \prod_{i} p_i^{k_i}$, we define:
\[
\psi(m) = \sum_{i} \psi_{\mathrm{pp}}(p_i, k_i).
\]

\medskip

\noindent\textbf{Proof Strategy.}
The proof splits naturally into two directions.

\textit{Forward direction ($m \in \mathrm{Ord}_N \Rightarrow \psi(m) \leq N$):}
If $M$ is an $N \times N$ integer matrix with order $m$, then $M$ satisfies
$M^m = I$, so its minimal polynomial divides $X^m - 1$. Since $M$ has exact order $m$,
the $m$-th cyclotomic polynomial $\Phi_m$ divides the minimal polynomial, and hence divides
the characteristic polynomial of $M$. By the degree constraint,
$\deg(\chi_M) = N \geq \deg(\Phi_m) = \varphi(m) \geq \psi(m)$.

\textit{Backward direction ($\psi(m) \leq N \Rightarrow m \in \mathrm{Ord}_N$):}
We construct an integer matrix of order $m$ with dimension exactly $\psi(m)$, which can
then be embedded into any larger dimension. The construction uses companion matrices of
cyclotomic polynomials. For a prime power $p^k$ with $\psi_{\mathrm{pp}}(p, k) > 0$,
the companion matrix of $\Phi_{p^k}$ is a $\varphi(p^k) \times \varphi(p^k)$ integer matrix
with order $p^k$. For composite $m$, we combine companion matrices for each prime power
factor in a block diagonal arrangement. The coprimality of orders ensures the combined
matrix has order $\mathrm{lcm}$ of the individual orders, which equals $m$.

\medskip

\noindent\textbf{Historical Context.}
The question of which orders can be achieved by integer matrices has a rich history.
The classical crystallographic restriction theorem states that rotations in two and
three dimensions preserving a lattice can only have orders 1, 2, 3, 4, or 6---a
fundamental result explaining the possible symmetries of crystals.

The general $N$-dimensional problem was studied by Kuzmanovich and
Pavlichenkov~\cite{KP2002}, who characterized achievable orders using a function
$W(m) = \sum (p_i - 1)p_i^{e_i - 1}$ equivalent to our $\psi$ function. Their
Theorem~2.7 establishes the bound using companion matrices of cyclotomic polynomials
and block diagonal constructions---the same techniques formalized here.

This formalization follows the exposition of Bamberg, Cairns, and
Kilminster~\cite{BCK2003}, who independently characterized achievable orders
and connected the result to permutation matrices and Goldbach's conjecture.

\begin{thebibliography}{KP2002}
\bibitem{BCK2003}
J.~Bamberg, G.~Cairns, and D.~Kilminster.
\newblock The crystallographic restriction, permutations, and {G}oldbach's conjecture.
\newblock \textit{Amer. Math. Monthly}, 110(3):202--209, 2003.

\bibitem{KP2002}
J.~Kuzmanovich and A.~Pavlichenkov.
\newblock Finite groups of matrices whose entries are integers.
\newblock \textit{Amer. Math. Monthly}, 109(2):173--186, 2002.
\end{thebibliography}

\chapter{The Psi Function}

The $\psi$ function measures the ``arithmetic complexity'' of a positive integer $m$.
For a prime power $p^k$, we define:
\begin{itemize}
\item $\psi_{\mathrm{pp}}(p,k) = \varphi(p^k)$ for $p$ odd or $k \geq 2$
\item $\psi_{\mathrm{pp}}(2,1) = 0$
\end{itemize}

For a general integer $m$ with prime factorization $m = \prod_i p_i^{k_i}$, we have:
$$\psi(m) = \sum_i \psi_{\mathrm{pp}}(p_i, k_i)$$

\inputleanmodule{Crystallographic.Psi.Basic}
\inputleanmodule{Crystallographic.Psi.Bounds}

\chapter{Integer Matrix Orders}

We define the set $\mathrm{Ord}_N$ of achievable orders for $N \times N$ integer matrices.

\inputleanmodule{Crystallographic.FiniteOrder.Basic}
\inputleanmodule{Crystallographic.FiniteOrder.Order}

\chapter{Companion Matrices}

Companion matrices provide a key construction for achieving specific orders via cyclotomic
polynomials.

\section{Definition and Basic Properties}

\inputleannode{companion-def}

The characteristic polynomial of the companion matrix is exactly the defining polynomial,
and by the Cayley-Hamilton theorem, $C_p$ satisfies $p(C_p) = 0$. This makes companion
matrices ideal for constructing matrices with prescribed minimal polynomials.

\inputleannode{lem-companion-aeval-zero}

\inputleannode{thm-companion-charpoly}

\inputleannode{thm-companion-pow-dvd}

\section{Cyclotomic Companion Matrices}

For achieving finite order $m$, we use the companion matrix of the $m$-th cyclotomic
polynomial $\Phi_m$. Since roots of $\Phi_m$ are primitive $m$-th roots of unity,
this companion matrix has order exactly $m$.

\inputleanmodule{Crystallographic.Companion.Cyclotomic}

\chapter{The Crystallographic Restriction Theorem}

The proof splits into two directions: showing $\psi(m) \leq N$ is necessary (forward)
and sufficient (backward) for $m \in \mathrm{Ord}_N$.

\inputleanmodule{Crystallographic.CrystallographicRestriction.Forward}
\inputleanmodule{Crystallographic.CrystallographicRestriction.Backward}
\inputleanmodule{Crystallographic.Main.MainTheorem}

\appendix
\chapter{Appendix}

This appendix collects technical lemmas used throughout the proof.
These are general-purpose results about finite sets, coprime products,
Euler's totient function, and matrix orders that support the main arguments.

\inputleanmodule{Crystallographic.Main.Lemmas}

\end{document}
