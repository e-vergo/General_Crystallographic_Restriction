\input{../../.lake/build/blueprint/library/Crystallographic}

\usepackage{amsmath, amsthm}
\usepackage{hyperref}

\theoremstyle{definition}
\newtheorem{definition}{Definition}[section]
\newtheorem{theorem}{Theorem}[section]
\newtheorem{proposition}{Proposition}[section]
\newtheorem{lemma}{Lemma}[section]
\newtheorem{corollary}{Corollary}[section]

\title{Crystallographic Restriction Theorem}

\newcommand{\Z}{\mathbb{Z}}
\newcommand{\N}{\mathbb{N}}

\begin{document}
\maketitle

\chapter{Introduction}

The \emph{Crystallographic Restriction Theorem} characterizes exactly which rotation orders
are achievable by integer matrices. Specifically, an $N \times N$ integer matrix can have
finite order $m$ if and only if $\psi(m) \leq N$, where $\psi$ is a function based on
Euler's totient function.

\chapter{Supporting Lemmas}

\inputleanmodule{Crystallographic.Lemmas}

\chapter{The Psi Function}

The $\psi$ function measures the ``arithmetic complexity'' of a positive integer $m$.
For a prime power $p^k$, we define:
\begin{itemize}
\item $\psi_{pp}(p,k) = \varphi(p^k)$ for $p$ odd or $k \geq 2$
\item $\psi_{pp}(2,1) = 0$
\end{itemize}

For a general integer $m$ with prime factorization $m = \prod_i p_i^{k_i}$, we have:
$$\psi(m) = \sum_i \psi_{pp}(p_i, k_i)$$

\inputleanmodule{Crystallographic.Psi.Basic}
\inputleanmodule{Crystallographic.Psi.Bounds}

\chapter{Integer Matrix Orders}

We define the set $\mathrm{Ord}_N$ of achievable orders for $N \times N$ integer matrices.

\inputleanmodule{Crystallographic.FiniteOrder.Basic}
\inputleanmodule{Crystallographic.FiniteOrder.Order}

\chapter{Companion Matrices}

Companion matrices provide a key construction for achieving orders via cyclotomic polynomials.

\inputleanmodule{Crystallographic.Companion}
\inputleanmodule{Crystallographic.Companion.Cyclotomic}

\chapter{The Crystallographic Restriction}

The proof splits into two directions: showing $\psi(m) \leq N$ is necessary (forward)
and sufficient (backward) for $m \in \mathrm{Ord}_N$.

\inputleanmodule{Crystallographic.CrystallographicRestriction.Forward}
\inputleanmodule{Crystallographic.CrystallographicRestriction.Backward}

\chapter{Main Theorem}

\inputleanmodule{Crystallographic.MainTheorem}

\end{document}
