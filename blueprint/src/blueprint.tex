\input{../../.lake/build/blueprint/library/Crystallographic}

\usepackage{amsmath, amsthm}
\usepackage{hyperref}

\theoremstyle{definition}
\newtheorem{definition}{Definition}[section]
\newtheorem{theorem}{Theorem}[section]
\newtheorem{proposition}{Proposition}[section]
\newtheorem{lemma}{Lemma}[section]
\newtheorem{sublemma}{Sublemma}[section]
\newtheorem{corollary}{Corollary}[section]
\newtheorem{remark}{Remark}[section]

\title{Crystallographic Restriction Theorem}

\newcommand{\Z}{\mathbb{Z}}
\newcommand{\N}{\mathbb{N}}

\begin{document}
\maketitle

\chapter{Introduction}

The \emph{Crystallographic Restriction Theorem} characterizes exactly which rotation orders
are achievable by integer matrices. Specifically, an $N \times N$ integer matrix can have
finite order $m$ if and only if $\psi(m) \leq N$, where $\psi$ is a function based on
Euler's totient function.

The project repository is \url{https://github.com/your-username/General_Crystallographic_Restriction}.

\chapter{The Psi Function and Main Definitions}

The main theorem requires the definition of the $\psi$ function, which measures the
``arithmetic complexity'' of a positive integer $m$. For a prime power $p^k$, we define:
\begin{itemize}
\item $\psi_{pp}(p,k) = \varphi(p^k)$ for $p$ odd or $k \geq 2$
\item $\psi_{pp}(2,1) = 0$
\end{itemize}

For a general integer $m$ with prime factorization $m = \prod_i p_i^{k_i}$, we have:
$$\psi(m) = \sum_i \psi_{pp}(p_i, k_i)$$

\inputleanmodule{Crystallographic.Definitions.Psi}

\chapter{Integer Matrix Orders}

\inputleanmodule{Crystallographic.Definitions.IntegerMatrixOrder}

\chapter{The Main Theorem}

\inputleanmodule{Crystallographic.MainTheorem}

\chapter{Companion Matrices}

\inputleanmodule{Crystallographic.Definitions.CompanionMatrix}
\inputleanmodule{Crystallographic.Proofs.CompanionMatrix}

\chapter{Lower Bound on Psi}

\inputleanmodule{Crystallographic.Proofs.PsiLowerBound}

\chapter{Rotation Matrices}

\inputleanmodule{Crystallographic.Definitions.RotationMatrices}
\inputleanmodule{Crystallographic.Proofs.RotationMatrices}

\chapter{Crystallographic Restriction Proof}

\inputleanmodule{Crystallographic.Proofs.CrystallographicRestriction}
\inputleanmodule{Crystallographic.ProofOfMainTheorem}

\end{document}
